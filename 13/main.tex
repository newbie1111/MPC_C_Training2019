%------------------------------------- ページサイズなどの書式設定
%¥documentclass[a4j,twocolumn, dvipdfmx]{jsarticle} % 二段組の構成にする
%¥documentclass[a4j,notitlepage]{jsarticle} % タイトルだけのページを作らない
\documentclass[a4j,titlepage,dvipdfmx]{jsarticle}   % タイトルだけのページを作る
%-------------------------------------コマンド定義
%styファイルのパスの簡略化
\newcommand{\stypath}{./sty}
%コードファイルの簡略化(./code/04のように毎回変更する)
\newcommand{\codepath}{./code/}
%記事ファイルの簡略化(codepathと同様)
\newcommand{\articlepath}{./article}
%------------------------------------- パッケージ読み込み
\usepackage[ipaex]{pxchfon}
%\usepackage{itembkbx}
\usepackage{\stypath/listings}
\usepackage{ascmac}
\usepackage{\stypath/jlisting}
\usepackage[dvipdfmx]{hyperref}%ハイパリンク
\lstset{% 
showstringspaces=false,%空白文字削除
language={C},% %言語選択
basicstyle={\upshape},% %標準の書体
identifierstyle={\small},% %キーワードでない文字の書体
ndkeywordstyle={\small},% %キーワードその2の書体
stringstyle={\small\ttfamily},% %””で囲まれた文字などの書体
frame={tb},% %枠、デザインなど
breaklines=true,% %行が長くなった時の自動改行
columns=[l]{fullflexible},% %書体による列幅の違いを調整するか
numbers=left,% %行番号を表示するか
xrightmargin=0zw,% %余白の調整?
xleftmargin=0zw,% %余白の調整
numberstyle={\scriptsize},%行番号の書体
stepnumber=1,% %行番号をいくつ飛ばしで表示するか
numbersep=1zw,% %行番号と本文の間隔
morecomment=[l]{//}% 
} 
\title{C言語講座第 13回}%何回か書き直す
\author{MPC部員}
\date{2019年11月21日}%日付も書き直す
\begin{document}
\maketitle
\section{前回の復習}
前回はmallocなどを使った動的確保を行いました。
pythonでの動的確保は今回やるリストの方が重要なので復習は飛ばします。
その分リストについていろいろやっていきましょう。




\section{list}
ついにデータ構造という範囲に入っていきます。\\
データ構造について\\
データ構造は簡単に言えばデータの管理方法(格納の仕方)です。\\
ソフトウェアの開発においてのデータ構造はプログラムのアルゴリズムの効率に大きく影響があります。\\
そのため、この範囲の説明はじっくりやっていきますので、わからないところはどんどん聞いてみてください。\\
講座ではリスト、キュー、スタックの3つを扱います。この3つは授業で扱うと思いますが結構複雑なので予習があるとかなり楽ができます。\\
今回はリスト構造についてのみやります。何故リストかというとpythonでの配列ではリストを使っているからです。\\
\subsection{リストとは}
データが順序付けられて並んだデータ構造のこと\\
自己参照構造体(ノード)とリスト型構造体(リスト)という2つの構造体により実現される\\
自己参照型ポインタ:本来のデータのほかに、自分自身と同じ構造体を指すポインタを持つ構造体のこと\\
言葉だけではよくわからないのでここで一度ソースを貼ります。\\
書き写すかぱっと見てから下を読み進めてください。
\lstinputlisting{\codepath/list19.c}
読んでもよくわかりませんが図でみると少しわかりやすいです。
\begin{table}[htb]
\begin{center}
\begin{tabular}{|c|c|c|c|c|c|c|c|c|c|}
\cline{1-1}\cline{2-2}\cline{4-4}\cline{6-6}\cline{8-8}\cline{10-10}
アドレス & 0 && 1 && 2 && 3 && 4 \\ \cline{1-1}\cline{2-2}\cline{4-4}\cline{6-6}\cline{8-8}\cline{10-10}
要素 & 要素1 && 要素2 && 要素3 && 要素4 && 要素5\\ \cline{1-1}\cline{2-2}\cline{4-4}\cline{6-6}\cline{8-8}\cline{10-10}
次のアドレス & *1 && *2 && *3 && *4 && null   \\ 
\cline{1-1}\cline{2-2}\cline{4-4}\cline{6-6}\cline{8-8}\cline{10-10}
\end{tabular}
\end{center}
\end{table}
このように配列に似た機能を持っています。ではなぜリストがあるのかです。\\
配列とリストの違い\\
配列は今まで通り、添え字を持った箱の連続のようなものです。\\
対してリストは要素をっ持った箱を任意でつなげたようなものです\\。
なので挿入の関数・削除の関数をつくることで箱と箱の間に要素の入った箱を追加したり、削除したりすることが簡単にできます。\\

ここで挿入と削除について書いた2例を出します
\lstinputlisting{\codepath/list19_insert.c}

\lstinputlisting{\codepath/list19_delete.c}
例えばリストの3番目を削除するとアドレスは0-1-2-4となり、その後2-4の間に挿入するとします。\\
そうすると以下のようになります。
\begin{table}[htb]
\begin{center}
\begin{tabular}{|c|c|c|c|c|c|c|c|c|c|}
\cline{1-1}\cline{2-2}\cline{4-4}\cline{6-6}\cline{8-8}\cline{10-10}
アドレス & 0 && 1 && 2 && 6 && 4 \\ \cline{1-1}\cline{2-2}\cline{4-4}\cline{6-6}\cline{8-8}\cline{10-10}
要素 & 要素1 && 要素2 && 要素3 && 要素6 && 要素5\\ \cline{1-1}\cline{2-2}\cline{4-4}\cline{6-6}\cline{8-8}\cline{10-10}
次のアドレス & *1 && *2 && *6 && *4 && null   \\ 
\cline{1-1}\cline{2-2}\cline{4-4}\cline{6-6}\cline{8-8}\cline{10-10}
\end{tabular}
\end{center}
\end{table}
ここで簡単な問題です。\\
配列とリストの参照速度について\\
・n番目の参照の速さ\\
・データの挿入、削除の速さ\\
それぞれで配列とリストどちらが早いと思いますか?



\include{\articlepath/sakaki_exercise}
\include{\articlepath/papy_exercise}
\section*{参考文献}
\noindent
[1]昨年までの講座資料\newline
[2]\href{http://9cguide.appspot.com}{苦しんで覚えるC言語}\newline
[3]\href{https://yukicoder.me/}{yukicoder}\newline
[4]\href{http://www.c-tipsref.com/reference/string.html}{C言語関数辞典}
\end{document}
